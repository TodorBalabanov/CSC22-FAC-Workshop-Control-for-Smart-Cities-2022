%===============================================================================
% ifacconf.tex 2022-02-11 jpuente  
% Template for IFAC meeting papers
% Copyright (c) 2022 International Federation of Automatic Control
%===============================================================================
\documentclass{ifacconf}

\usepackage{graphicx}
\usepackage{natbib} 

\begin{document}
\begin{frontmatter}

\title{Android Content Providers in Mobile Distributed Computing \thanksref{footnoteinfo}} 

\thanks[footnoteinfo]{Velbazhd Software LLC}

\author[First]{Gergana Mateeva} 
\author[First]{Dimitar Parvanov} 
\author[Second]{Ioan Dimitrov} 
\author[First]{Iliyan Iliev}
\author[First]{Todor Balabanov} 

\address[First]{Bulgarian Academy of Sciences\\Institute of Information and Communication Technologies\\acad. Georgi Bonchev Str., block 2, 1113 Sofia, Bulgaria\\(gergana.mateeva, dimitar.parvanov, iliyan.iliev, todor.balabanov) @iict.bas.bg}
\address[Second]{Technical University of Sofia\\Faculty of Electronic Engineering and Technology\\8 St. Kliment Ohridski Blvd., block 1, 1756 Sofia, Bulgaria\\joancdimitrov@tu-sofia.bg}

\begin{abstract}
Distributed computing and volunteer computing became very popular in the last two decades. Both are used for problems easily separable for simultaneous calculations on many heterogeneous machines. The only difference is that volunteer computing has been done on a donated calculating power. With the rise of smart mobile devices, volunteer computing appeared in the world of mobile distributed computing. With its capabilities, Android OS became an attractive environment for such computations. Android's content providers became a valuable tool for information transfer in mobile distributed computing applications. 
\end{abstract}

\begin{keyword}
Android, distributed computing, volunteer computing
\end{keyword}

\end{frontmatter}

\section{Introduction}

\section{}

\subsection{}

\section{Conclusion}

\begin{ack}
This research is funded by Velbazhd Software LLC and it is partially supported by the Bulgarian National Science Fund by the project “Mathematical models, methods and algorithms for solving hard optimization problems to achieve high security in communications and better economic sustainability”, KP-06-N52/7/19-11-2021.
\end{ack}

\bibliography{ifacconf}            

\end{document}
